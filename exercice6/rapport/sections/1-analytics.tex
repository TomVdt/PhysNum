\section{Analytical results}

The theory behind this system as well as the theory behind the Crank-Nicolson method is shown in the numerical physics book \cite{physnumbook} in section 4.3. Some implementation details will be shown here.

The studied system is in 1 dimension located between \(x_L\) and \(x_R\) (\(x_L < x_R\)), with a potential given by:
\begin{equation}
    V(x) = \frac{1}{2} V_0 \left( 1 + \cos \left( 2\pi n_V \frac{x - x_L}{x_R - x_L} \right) \right)
\end{equation}
where \(V_0\) and \(n_V\) are given for the simulations. We define \(L = x_R - x_L\) the length of the potential well. The system is initialised as a Gaussian wave packet:
\begin{equation}
    \psi(x, t=0) = C \exp(i k_0 x)\exp\left(-\frac{(x-x_0)^2}{2\sigma^2}\right)
\end{equation}
where $k_0$ is the wave number defined as $k_0 = 2\pi n/L$, $x_0$ is the center of the packet and \mbox{$\sigma = \sigma_{norm}L$} is its standard deviation. For the simulations, $n$, $x_0$ and $\sigma_{norm}$ will be given. We find that the wavelength is $\lambda = 2\pi/k_0 = L/n$ and the standard deviation of the packet is also proportional to the length. This gives a simulation scaled to the length of the potential well. Quantum mechanics require normalised states so \(C\) is a constant such that \mbox{\(\int_{x_L}^{x_R}|\psi(x,0)|^2 dx = 1\)}, i.e.
% \begin{equation}
%     C = \left. 1\middle/\sqrt{\int_{x_L}^{x_R} \left|\exp(i k_0 x)\exp\left(-\frac{(x-x_0)^2}{2\sigma^2}\right)\right|^2 \mathrm{d}x} \right.
% \end{equation}
\begin{equation}
    C = \frac{1}{\sqrt{\int_{x_L}^{x_R} \left|\exp(i k_0 x)\exp\left(-\frac{(x-x_0)^2}{2\sigma^2}\right)\right|^2 \mathrm{d}x}}
    \label{eq:c_norm}
\end{equation}
calculated numerically at the beginning of the simulation.

The numerical simulation requires an uniform mesh $\{x_i\}_{i=0}^N$ for the positions where $N$ is the number of intervals in the mesh. The simulation takes $N$ and calculates the space between each point $\Delta x = L/N$. The constant timestep $\Delta t$ is also given.

Every intergal presented here will be calculated calculated using the trapezoïdal method, where \(\{x_i\}\), \(i = 1, \dots, N+1\), is the uniformly discretized interval between \(x_a\) and \(x_b\) given by the mesh, and where \(N\) is the number of intervals:
\begin{equation}
    \int_{x_a}^{x_b} f(x) \dd x = \frac{x_b - x_a}{N} \sum_{i=1}^N \frac{f(x_i) + f(x_{i+1})}{2}
\end{equation}
The method used for derivation is finite differences. For first derivatives this is given by:
\begin{equation}
    \frac{\partial \psi}{\partial x}(x) = \frac{\psi(x+\Delta x) - \psi(x - \Delta x)}{2\Delta x}
\end{equation}
and for the border cases we take forward and backward finite differences. For second derivatives we have:
\begin{equation}
    \frac{\partial^2 \psi}{\partial x^2}(x) = \frac{\psi(x+\Delta x) - 2\psi(x) + \psi(x - \Delta x)}{\Delta x^2}
    \label{eq:second_derivative}
\end{equation}
and border cases use the same formula, with $\psi \equiv 0$ outside of $[x_L,x_R]$, if not mentioned otherwise.

For the time evolution the Crank-Nicolson method corresponding to Equation (4.90) in the numerical physics book \cite{physnumbook} is implemented. For this three matrices are implemented similarly: the hamiltonian $\mathbf{H}$ and the matrices $\mathbf{A}$ and $\mathbf{B}$ from Equation (4.99) of the book \cite{physnumbook} giving:
\begin{equation}
    \mathbf{A} \Psi(t+\Delta t) = \mathbf{B} \Psi(t)
\end{equation}
with $\Psi$ the vector of the values of $\psi$ for every point $x_i$ of the mesh. The matrices $\mathbf{A}$ and $\mathbf{B}$ correspond respectively to $\mathbf{I} \pm (i/\hbar)(\Delta t/2) \mathbf{H}$ with $\mathbf{I}$ the identity. We thus need to find the implementation for $\mathbf{H}$ first. It is given that the corresponding operator is:
\begin{equation}
    \hat{\mathrm{H}}(x) = -\frac{\hbar^2}{2m}\frac{\partial^2}{\partial x^2} + V(x)
\end{equation}
By applying the method from \autoref{eq:second_derivative} we know that the matrix representation will only be tridiagonal and we find:
\begin{equation}
    \mathbf{H} = \left( \begin{matrix}
        dH_0 & cH_0      &          &   \\
        aH_0 & \ddots    &          &   \\
             &           & \ddots   & cH_{N-1} \\
             &           & aH_{N-1} & dH_N
        \end{matrix} \right)
        , \qquad 
        \begin{cases} aH_i = cH_i = -\alpha &, \forall i \\
        &\\
        dH_i = 2\alpha + V(x_i) &, \forall i
        \end{cases}
        \label{eq:H_matrix}
\end{equation}
with $\alpha = \hbar^2/(2m\Delta x^2)$. We take the same representation for $\mathbf{A}$ and $\mathbf{B}$:
\begin{equation}
    A = \left( \begin{matrix}
        dA_0 & cA_0      &          &   \\
        aA_0 & \ddots    &          &   \\
             &           & \ddots   & cA_{N-1} \\
             &           & aA_{N-1} & dA_N
        \end{matrix} \right)
    ,\qquad
    B = \left( \begin{matrix}
        dB_0 & cB_0      &          &   \\
        aB_0 & \ddots    &          &   \\
             &           & \ddots   & cB_{N-1} \\
             &           & aB_{N-1} & dB_N
        \end{matrix} \right)
\end{equation}

TODO C EST MOCHE FAIRE UNE JOLIE EQUATION
Using the formula for $\mathbf{A}$ and $\mathbf{B}$ we find $aA_i = -aB_i = -a , \forall i \in\{1, \dots, N-2\}$ and $dA_i = 1 + 2a + b, dB_i = 1 - 2a - b , \forall i \in\{1, \dots, N-1\}$ with $a = (i\hbar \Delta t)/(4m \Delta x^2)$ and $b = (i\Delta t V)/(2\hbar)$

The border conditions imposed \(\psi(x_L, t) = \psi(x_R, t) = 0\,\, \forall t\). This was implemented by setting the coefficients of \(A\) and \(B\) matrices such that: FEUR CA ME SAOULE

A few observables are implemented to be returned by the simulation. First, the probability to find the particle with an observation between two given points $x_a$ and $x_b$ is implemented as:
\begin{equation}
    \mathrm{P}(x\in[x_a,x_b]) = \int_{x_a}^{x_b}|\psi(x,t)|^2 dx = \int_{x_a}^{x_b}\psi^*(x,t)\psi(x,t) dx
\end{equation}

The other observables are computed according to the discretisation formulas. These observables are:
\begin{itemize}
    \item Energy of particle: \(E(t) = \qavg{H}(t) = \int_{x_L}^{x_R} \psi^*(x,t) H(x) \psi(x,t) \dd x\)
    \item Average position: \(\qavg{x}(t) = \int_{x_L}^{x_R} \psi^*(x,t) x \psi(x,t) \dd x\)
    \item Average \(x^2\): \(\qavg{x^2}(t) = \int_{x_L}^{x_R} \psi^*(x,t) x^2 \psi(x,t) \dd x\)
    \item Average momentum: \(\qavg{p}(t) = \int_{x_L}^{x_R} \psi^*(x,t) \left( -i \hbar \frac{\partial \psi(x,t)}{\partial x} \right) \dd x\)
    \item Average \(p^2\): \(\qavg{p^2}(t) = \int_{x_L}^{x_R} \psi^*(x,t) \left( - \hbar^2 \frac{\partial^2 \psi(x,t)}{\partial x^2} \right) \dd x\). We consider the second derivatives at the borders to be \(0\)
\end{itemize}
