\section{Conclusion}
The behaviour of a quantum particle was analysed in various situations. First many characteristics of the simulated system were studied in a well known case, the particle trapped in an infinite well, to verify the coherence of the simulation with the principles of quantum mechanics. Once this was established simulations of quantum tunneling were run to illustrate this surprising behaviour of a quantum particle with no equivalent in classical physics. Characteristics of the simulation in this case were also determined as well as the general behaviour for various energies. We can conclude that the Crank-Nicolson method gives good results when applied to quantum mechanics and allows to study its phenomena.

