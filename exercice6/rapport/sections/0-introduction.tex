\section{Introduction}

The behavior of quantum particles can be hard to predict, certainly when considering some properties of the particle can't be observed, like the imaginary and real part of the wave function. However, the Schrödinger equation is quite easy to simulate. The Crank-Nicolson method is one way to carry out these simulations. In this report we will study how a quantum particle "moves", under different potentials. We will first look at a particle inside an infinite potential well and verify its properties, such as its average position, energy, wave function and more. This particle will be compared to a classical particle sharing some properties. Heisenberg uncertainty principle will also be verified. Afterwards, the quantum particle will be simulated in a double potential well. The convergence of the algorithm, the Crank-Nicolson method, will be studied. Finally, the tunnel effect will be shown.
