\section{Introduction}

The behavior of quantum particles can be hard to predict, especially when considering that some properties of the particle can't be observed, like the wave function. However, the Schrödinger equation is quite easy to integrate numerically. The Crank-Nicolson method is one way to carry out these simulations. In this report we will study how a quantum particle "moves", under different potentials. We will first look at a particle inside an infinite potential well and verify its properties, such as its average position, energy, wave function and more. This particle will be compared to a classical particle sharing some of its properties. Heisenberg's uncertainty principle will also be verified. Afterwards, the quantum particle will be simulated in a double potential well allowing to illustrate quantum tunneling. The convergence of the Crank-Nicolson algorithm will be studied in this case.
