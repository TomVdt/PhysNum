\section{Analytical results}

For the purpose of this study it is first needed to prove some analytical results that will later be achieved through numerical simulations.

We are considering a sphere (tennis ball) of mass $m$ and radius $R$. It is rotating according to $\vec{\omega} = \omega \vec{e_z}$. It is moving in the gravity field $\vec{g} = -g\vec{e_y}$ and inside a fluid of density $\rho$ which applies a force due to the Magnus effect:
\be
    \vec{F}_p = \mu R^3 \rho \vec{\omega} \times \vec{v}
    \label{eq:Magnus}
\ee

We want to determine the movement of the ball knowing the initial velocity $\vec{v_0}$ and rotation $\vec{\omega}$. In an effort to simplify the problem we assume that the rotation is constant and we consider trajectories occuring only on the $(x,y)$ plane.


\subsection{System of differential equations}
\underline{Question 1.1-(a)}

Let us take the vector: $\textbf{y} = (x,y,v_x,v_y)$

We are searching for $f$ such that:
\be
    \frac{\dd \textbf{y}}{\dd t} = \left(\begin{matrix}{l} v_x \\ v_y \\ a_x \\ a_y \end{matrix}\right) = f(\textbf{y})
    \label{eq:a_question}
\ee

We know that $m\vec{a} = \vec{F}_p + m\vec{g}$ and from \autoref{eq:Magnus} we have:
\[\vec{F}_p = \mu R^3 \rho \omega \left(\begin{matrix}{l} -v_y \\ v_x \\ 0 \end{matrix}\right)\]

We get from this and $g_z = 0$ that $a_z = 0$ so the trajectory will indeed stay inside the $(x,y)$ plane for $v_z(0) = 0$ and $z(0) = 0$. Most importantly we also get: 
\[ \left(\begin{matrix}{l} a_x \\ a_y\end{matrix}\right) = 
    \left( \begin{matrix}{l} -\frac{\mu R^3 \rho \omega}{m} v_y \\ \frac{\mu R^3 \rho \omega}{m} v_x - g \end{matrix}\right) \]

Which allows us to rewrite \autoref{eq:a_question} as:
\be
    \frac{\dd \textbf{y}}{\dd t} = f(\textbf{y}) = \left(\begin{matrix}{l}
    v_x \\
    v_y \\
    -\frac{\mu R^3 \rho \omega}{m} v_y \\
    \frac{\mu R^3 \rho \omega}{m} v_x - g
    \end{matrix}\right)
    \label{eq:ode}
\ee

\subsection{Mechanical energy}
\underline{Question 1.1-(b)}

We are searching for the total mechanical energy of the system. We know that for a rigid body such as the tennis ball considered in this problem the total mechanical energy is the sum of: the translational kinetic energy, the rotational kinetic energy and the potential so:
\[ 
    \mathrm{E_{tot}} = \frac{1}{2}mv^2 + \frac{1}{2}I\omega^2 + mgy 
\]

We get the moment of inertia of a sphere for any rotation around an axis going through the center: $I = \frac{2}{5}mR^2$ \cite*{moment-inertia}.

Which yields the final formula for the energy:
\be
    \mathrm{E_{tot}} = \frac{1}{2}mv^2 + \frac{1}{5}mR^2\omega^2 + mgy
\ee

We now want to know if this energy is conserved. We know that the gravitational force from the $\vec{g}$ field is conservative. We now consider the force $\vec{F}_p$. It is strictly orthogonal to the velocity at any point from \autoref{eq:Magnus} which means that $\vec{F}_p$ does not produce any work throughout the movement. We can conclude that we have no non-conservative forces doing work so the total mechanical energy must be conserved.

\subsection{Zero-gravity situation}
\underline{Question 1.1-(c)}

All the considerations in this subsection will be done inside the $(x,y)$ plane since we have shown that the movement always stays inside of it. The $z$ and $v_z$ coordinates are thus always 0.

We now want to solve \autoref{eq:ode} for $\vec{g}=0$ and the initial conditions $\vec{x}(0)=0$ and $\vec{v}(0)=v_0\vec{e_x}$. Let us write $\alpha = \frac{\mu R^3 \rho \omega}{m}$. We have:
\begin{eqnarray}
    a_x = \dot{v_x} = -\alpha v_y \\
    a_y = \dot{v_y} = \alpha v_x \label{eq:nograv:ay}
\end{eqnarray}

Taking the time derivative yields:
\be
    \ddot{v_x} = -\alpha \dot{v_y} = -\alpha ^2 v_x \\
    \implies \ddot{v_x} + \alpha^2 v_x = 0
\ee
which we recognise as the harmonic oscillator equation. Its general solution is:
\be
    v_x(t) = A\cos(\alpha t + \varphi)
    \label{eq:general_vx}
\ee

We can deduce the general solution for \(x(t)\) from this:
\be
    x(t) = \frac{A}{\alpha}\sin(\alpha t + \varphi) + c_1
    \label{eq:general_x}
\ee

We can find \(y(t)\) by substituing \(v_x\) into \autoref{eq:nograv:ay}
\begin{eqnarray}
    \dot{v_y}(t) = \alpha v_x = A\alpha\cos(\alpha t + \varphi) \nonumber\\
    \implies v_y(t) = A \sin(\alpha t + \varphi) + c_2
    \label{eq:general_vy} \\
    \implies y(t) = -\frac{A}{\alpha}\cos(\alpha t + \varphi) + c_2 t+ c_3
    \label{eq:general_y}
\end{eqnarray}

Using the mentioned initial conditions inside these equations gives us 5 unknowns for 4 equations. For symmetry reasons we can set $\varphi = 0$ which gives the other constants:
\[ c_1 = 0, \quad c_2 = 0, \quad c_3 = \frac{v_0}{\alpha}, \quad A = v_0\]

We now want to characterise the type of movement we are having with this ball. We show that the acceleration is orthogonal to the velocity and has a constant norm:
\begin{gather}
    \vec{a} \cdot \vec{v} = \alpha(-v_y v_x + v_x v_y) = 0 \\
    ||\vec{a}||^2 = \alpha^2 ((-v_y)^2 + (v_x)^2) = \alpha^2 A^2 = \mathrm{cst}
\end{gather}

This means that the movement is circular uniform and we can write:
\begin{gather} 
    a = \frac{v^2}{r} , \quad a = \alpha v_0, \quad v = \mathrm{cst} = v_0 \nonumber\\
    \Rightarrow r = \frac{v_0}{\alpha}
\end{gather}
with $r$ the radius of the circular movement. The angular frequency is given by: \mbox{$\Omega = \frac{v}{r} = \alpha$}.

For clarity we rewrite \autoref{eq:general_vx} to \ref{eq:general_y} with the constants found and reminding that \mbox{$\alpha = \frac{\mu R^3 \rho \omega}{m}$}:
\be
    \begin{cases}
        v_x(t) = v_0\cos(\alpha t) \\
        % \label{eq:vx_g0}
        v_y(t) = v_0\sin(\alpha t) \\
        % \label{eq:vy_g0}
        x(t) = \frac{v_0}{\alpha}\sin(\alpha t) \\
        % \label{eq:x_g0}
        y(t) = \frac{v_0}{\alpha}(1 - \cos(\alpha t)) \\
    \end{cases}
    \label{eq:y_g0}
\ee


\subsection{Gravity with no initial speed situation}
\underline{Question 1.1-(d)}

We now want to solve the same problem with $g\ne0$. The given initial conditions are $\vec{x}(0)=0$ and $\vec{v}(0)=0$.

We choose a new referential $\mathcal{R}'$ moving at a constant velocity $\vec{v}_E = \frac{m}{\omega^2 \mu R^3 \rho}\vec{\omega}\times\vec{g} = \frac{1}{\alpha}\vec{e}_z\times\vec{g}$ relative to our first referential $\mathcal{R}$.


We first write down the equations of motion in the new referential using $\vec{v} = \vec{v}' + \vec{v}_E \Rightarrow \vec{a} = \vec{a}'$:
\begin{align*}
    \vec{a}' &= \frac{1}{m}\vec{F}_p + \vec{g} \\
    &= \alpha \vec{e}_z\times\vec{v} + \vec{g} \\
    &= \alpha \vec{e}_z\times\vec{v}' + \alpha \vec{e}_z\times\vec{v}_E + \vec{g} \\
    &= \alpha \vec{e}_z\times\vec{v}' + \vec{e}_z\times(\vec{e}_z\times\vec{g}) + \vec{g} \\
    &= \alpha \vec{e}_z\times\vec{v}' + g(-\vec{e}_z\times(\vec{e}_z\times\vec{e}_y) - \vec{e}_y) \\
    &= \alpha \vec{e}_z\times\vec{v}' + g(\vec{e}_y - \vec{e}_y) \\
\end{align*}
\be
    \implies \vec{a}' = \alpha \left(\begin{matrix}{l} -v_y' \\ v_x' \\ 0 \end{matrix}\right)
    \label{eq:motionRprime}
\ee

In the question \underline{1.1-(c)} we had an equation of motion $\vec{a} = \alpha \vec{e}_z\times\vec{v}$ which has the exact same form and thus the general solutions are the same for $\vec{a}'$ and $\vec{v}'$ as in \autoref{eq:general_vx} to \ref{eq:general_y}.

Assuming like before $\varphi = 0$ and using the given initial conditions we can find:
\[ A = -v_E, \quad c_3 = -\frac{v_E}{\alpha}, \quad c_1 = 0, \quad c_2 = 0 \]
with $\vec{v_E} = \frac{g}{\alpha} \vec{e}_x$.

This gives us our final solution by changing back to the $\mathcal{R}$ referential knowing it will have translated along $\hat{x}$ by $x = x' + v_E t$:
\be
    \begin{cases}
        x(t) = -\frac{g}{\alpha^2}\sin(\alpha t) + \frac{g}{\alpha}t \\
        y(t) = \frac{g}{\alpha^2}(\cos(\alpha t) - 1)
    \end{cases}
    \label{eq:x_fall}
    % \label{eq:y_fall}
\ee

We want to determine if the ball will at some point get back to its starting height $y = 0$. We see easily that the movement according to $y$ is periodic with $\cos$ being the only term depending on $t$. More formally, because $\cos$ is $2\pi$-periodic, we have for $t_\mathrm{fin} = 2\pi / \alpha$
\[ y(t_\mathrm{fin}) = \frac{g}{\alpha^2}\left(\cos\left(\frac{2\pi \alpha}{\alpha}\right) - 1\right) = \frac{g}{\alpha^2}(1 - 1) = 0 \]
which does show that at $t_\mathrm{fin} = 2\pi / \alpha = 2\pi m / \mu R^3\rho\omega$ the ball gets back to its original height. This happens at a distance $L$ from the origin:
\[
    L = x(t_\mathrm{fin}) = -\frac{g}{\alpha^2}\sin\left(\frac{2\pi \alpha}{\alpha}\right) + \frac{2\pi g}{\alpha^2}
\]
\be
    L = \frac{2\pi g}{\alpha^2} = \frac{2\pi g m^2}{(\mu R^3\rho\omega)^2}
\ee
