\section{Introduction}

In this paper we aim at giving a comprehensive numerical analysis of a typical case study, the trajectory of a rotating ball under the influence of the Magnus effect. This analysis will be done using three related types of algorithms: Euler explicit, implicit and semi-implicit. Their numerical convergence and stability will be studied in the various cases where they have been used.

The algorithms used will be implemented through a C++ simulation and the obtained data analysed using classic python libraries. The first simulation will show what happens if a rotating ball is given an initial velocity in a zero gravity environment. The second simulation will have two different iterations of the same situation, a rotating ball being released with zero velocity in a gravity field. The first iteration will show how, when not accounting for air drag, the ball eventually goes back to its original height and enters a periodic process. The second will account for air drag and analyse the deflected trajectory. Some analytical results will also be proven in order to verify the soundness of our simulation by comparing them to the computational results.
