\section{Simulations}

\subsection{Rotating ball in zero gravity}

The simulations were done on a tenis ball of mass $m = 0.056$ kg and radius $R = 0.033$ m, with air density $\rho = 1.2$ kg/m$^3$. The coefficient $\mu = 6$ has been chosen. The ball is sent with initial conditions $\omega = 10$ rotations/s ($\omega = 20\pi$ rad/s), $\vec{x}(0) = \vec{0}$, $\vec{v}(0) = 5 \vec{e_x}$ and its trajectory is simulated until \mbox{$t_\textrm{fin} = 60$ s}.

To lighten the notation, we will use EE for the Explicit Euler method, IE for Implicit Euler and SE for Semi-implicit Euler.

The position after $t_\textrm{fin}$ is shown in \autoref{fig:nograv:trajectory_all}. We can see that the EE method overshoots the expected trajectory, while the IE method undershoots it and SE remains almost exactly on the analytical result.
\begin{wrapfigure}{hR}{0.45\linewidth}
    \centering
    \includegraphics[width=0.85\linewidth]{figures/nograv_trajectory_all.pdf}
    \caption{Postion of the ball after $t_\textrm{fin}$ for different methods ($n_\textrm{steps} = 2000$)}
    \label{fig:nograv:trajectory_all}
    \vspace*{-1cm}
\end{wrapfigure}


\subsubsection{Numeric convergence}

The numeric convergence of the final position of each method is illustrated in \autoref{fig:nograv:convergence}. We observe that the error on the final position tends to 0 as $\Delta t \rightarrow 0$ for every method, meaning that they all converge numerically. Furthermore, the convergence order is given by the slope of the line passing through the points. We can thus deduce that the convergence order is 1 for EE and IE, and 2 for SE.

\begin{figure}[h]
    \centering
    \begin{subfigure}{0.45\linewidth}
        \centering
        \includegraphics[width=\linewidth]{figures/nograv_numeric_convergence_all.pdf}
        \caption{Error on final position w.r.t. analytical solution. $n_\textrm{steps}$ was varied between 100 and 4000.}
        \label{fig:nograv:convergence}
    \end{subfigure}
    \hspace*{0.2cm}
    \begin{subfigure}{0.45\linewidth}
        \centering
        \includegraphics[width=\linewidth]{figures/nograv_energy_error_all.pdf}
        \caption{Error on final energy for each method where the error is defined as $\max(E_\textrm{mec}) - \min(E_\textrm{mec}).$ (maximum $n_\textrm{steps}=4000$)}
        \label{fig:nograv:energy_error}
    \end{subfigure}
    \caption{Numeric convergence analysis for different methods}
\end{figure}

The numeric convergence of the energy for each method is also given in \autoref{fig:nograv:energy_error}. Again, we observe that the error clearly tends to 0, as $\Delta t \rightarrow 0$ for EE and IE. The error for SE remains quite constant and is of the order of $10^{-14}$, close to the limit for double-precision floating point numbers. We thus conclude that every method converges numerically. The convergence order for EE and IE is 1, but SE does not have a convergence order as it's energy remains pretty much constant for a wide range of $\Delta t$.

\subsubsection{Numeric stability}

To illustrate the numeric stabilty of each method, the energy over time has been plotted in \autoref{fig:nograv:energy}. EE is numerically unstable, as its energy over time grows exponentially for smaller $n_\textrm{steps}$. IE is stable, but loses energy over time, decaying exponentially. Using a curve fit of the function $g(t) = A + e^{\gamma t}$, where A is an offset and $\gamma$ is the growth rate, to calculate the growth or decrease rate we obtain the results presented in \autoref{tab:nograv:rate}. Finally, the SE method remains at constant energy even for smaller $n_\textrm{steps}$, meaning it is both stable and conserves energy.

\begin{figure}[h]
    \centering
    \begin{subfigure}{0.5\linewidth}
        \centering
        \includegraphics[width=\linewidth]{figures/nograv_energy_explicit.pdf}
        \caption{Explicit}
    \end{subfigure}%
    \begin{subfigure}{0.5\linewidth}
        \centering
        \includegraphics[width=\linewidth]{figures/nograv_energy_implicit.pdf}
        \caption{Implicit}
    \end{subfigure}
    \begin{subfigure}{0.5\linewidth}
        \centering
        \includegraphics[width=\linewidth]{figures/nograv_energy_semiimplicit.pdf}
        \caption{Semi-implicit}
    \end{subfigure}
    \caption{Energy over time for different methods and $n_\textrm{steps}$. To improve the results, the tolerance on the error was set to $10^{-10}$}
    \label{fig:nograv:energy}
\end{figure}

\begin{table}[h]
    \centering
    \begin{tabulary}{0.8\linewidth}{C|C|C|C|C}
        \toprule
        $n_\textrm{steps}$ & 200 & 400 & 1600 & 4000 \\
        \midrule
        EE & 0.0203 & 0.0096 & 0.0023 & 0.0009 \\
        IE & -0.0146 & -0.0081 & -0.0022 & -0.0009 \\
        \bottomrule
    \end{tabulary}
    \caption{Rate of growth or decrease for non energy-conserving methods}
    \label{tab:nograv:rate}
\end{table}

\subsection{Rotating ball with gravity}
\subsubsection{Method comparison}
\label{seq:gravrot:comp}


Another set of simulations was run with $\vec{g} = -9.81 \vec{e}_y$ to analyse a more realistic trajectory. The tenis ball is let go at $\vec{x}(0) = \vec{0}$ and with $\vec{v}(0) = \vec{0}$. It rotates with the same angular speed $\omega$ as in the previous part. The simulation was run up to the time $t_\mathrm{fin}$ calculted in \autoref{seq:analytics_gravity} for the numerical values given. The given results for trajectories and energy are shown respectively in \autoref{fig:rotate_grav:traj} and \autoref{fig:rotate_grav:nrj}.

\begin{figure}[h]
    \centering
    \includegraphics[width=0.7\linewidth]{figures/rotate_grav_trajectories.pdf}
    \caption{Trajectories of the falling ball with each method}
    \label{fig:rotate_grav:traj}
\end{figure}

\begin{figure}[h]
    \centering
    \includegraphics[width=0.7\linewidth]{figures/rotate_grav_energy.pdf}
    \caption{Trajectories of the falling ball with each method}
    \label{fig:rotate_grav:nrj}
\end{figure}

% \begin{wrapfigure}{hR}{0.6\linewidth}
%     \vspace*{\fill}
%     \centering
%     {\includegraphics[width=\linewidth]{figures/rotate_grav_trajectories.pdf}
%     \subcaption{Trajectories of the falling ball with each method}
%     \label{fig:rotate_grav:traj}
%     }\par\vfill
%     {\includegraphics[width=\linewidth]{figures/rotate_grav_energy.pdf}
%     \subcaption{Evolution of the energy of the simulation for each method}
%     \label{fig:rotate_grav:nrj}
%     }
%     \caption{Simulation of the falling tennis ball with the three methods EE, IE and SE \mbox{($n_\textrm{steps}=2000$)}}
%     \label{fig:rotate_grav}
% \end{wrapfigure}

The trajectories obtained are extremely similar and diverge slightly around the end of the simulation as the close-up view shows. The expected behavior is observed with EE slightly overshooting, having gained energy, and IE falling short of reaching the original height of $y(0) = 0$ with a loss of energy. As before, the SE method is extremely precise, ending up exactly at the expected height at the time of the end of the simulation corresponding to the theoretical value.

In particular the \autoref{fig:rotate_grav:nrj} shows the great stability of the SE method with an energy seemingly constant. The EE and IE are more unstable with the energies diverging exponentially from the theoretical constant starting value. This can be illustrated by observing the trajectories for larger times as shown in \autoref{fig:rotate_grav:yolo} for $t_\mathrm{fin} = 300$. In this new simulation it is obvious that the EE diverges strongly. Observing the IE it can also be seen that its oscillations diminish through times and become imperceptible by the end of the simulation.

\begin{figure}[h]
    \centering
    \includegraphics[width=0.8\linewidth]{figures/rotate_grav_trajectories_yolo.pdf}
    \caption{Trajectories for the falling tennis ball with each method for $t_\mathrm{fin} = 300$}
    \label{fig:rotate_grav:yolo}
\end{figure}








\subsubsection{Adding drag force}

The aerodynamic drag force
\be
    \vec{F_t} = -\frac{1}{2}C_t \rho S \|v\| \vec{v}
\ee
was added to the simulation by adding a term to $f(\textbf{y})$ from \autoref{eq:ode}
\be
    f(\textbf{y}) = \left(\begin{matrix}{l}
    v_x \\
    v_y \\
    -\frac{\mu R^3 \rho \omega}{m} v_y  - \frac{1}{2m} C_t \rho S \|v\| v_x \\
    \frac{\mu R^3 \rho \omega}{m} v_x - g - \frac{1}{2m} C_t \rho S \|v\| v_y
    \end{matrix}\right)
\ee

In order to simulate this system, the SE method was used as it yields better results than EE or IE. The parameters of the simulation were as described in \ref{seq:gravrot:comp} and with $C_t = 0.35$, $S = \pi R^2$.

\begin{figure}[h]
    \centering
    \begin{subfigure}{0.49\linewidth}
        \centering
        \includegraphics[width=\linewidth]{figures/grav_frict_position.pdf}
        \caption{Trajectory after $t_\textrm{fin}$ ($n_\textrm{steps}=4000$)}
        \label{fig:gravfrict:pos}
    \end{subfigure}%
    \hspace*{0.2cm}
    \begin{subfigure}{0.45\linewidth}
        \begin{subfigure}{\linewidth}
            \centering
            \includegraphics[width=\linewidth]{figures/grav_frict_convergence_x.pdf}
        \end{subfigure}
        \begin{subfigure}{\linewidth}
            \centering
            \includegraphics[width=\linewidth]{figures/grav_frict_convergence_y.pdf}
        \end{subfigure}
        \caption{Convergence analysis for the final position. $n_\textrm{steps}$ was varied between 100 and 4000.}
        \label{fig:gravfrict:conv}
    \end{subfigure}
    \caption{Simulation of the system including friction using the SE method}
\end{figure}

The trajectory is shown in \autoref{fig:gravfrict:pos} and seems to behave as expected, with an equilibrium between the applied force that make it follow a straight path. As shown in \autoref{fig:gravfrict:conv}, the convergence order of the final position is $n=2$.

\subsection{Additionnal results}

Pour les extensions \& stuff
