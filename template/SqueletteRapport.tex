% debut d'un fichier latex standard
\documentclass[a4paper,12pt,twoside]{article}
% Tout ce qui suit le symbole "%" est un commentaire
% Le symbole "\" désigne une commande LaTeX

% pour l'inclusion de figures en eps,pdf,jpg, png
\usepackage{graphicx}

% quelques symboles mathematiques en plus
\usepackage{amsmath}

% le tout en langue francaise
\usepackage[french]{babel}

% on peut ecrire directement les caracteres avec l'accent
%    a utiliser sur Linux/Windows (! dépend de votre éditeur !)
%\usepackage[utf8]{inputenc} % 
%\usepackage[latin1]{inputenc} % Pour Kile
\usepackage[T1]{fontenc}

%    a utiliser sur le Mac ???
%\usepackage[applemac]{inputenc}

% pour l'inclusion de liens dans le document 
\usepackage[colorlinks,bookmarks=false,linkcolor=blue,urlcolor=blue]{hyperref}

\paperheight=297mm
\paperwidth=210mm

\setlength{\textheight}{235mm}
\setlength{\topmargin}{-1.2cm} % pour centrer la page verticalement
%\setlength{\footskip}{5mm}
\setlength{\textwidth}{16.5cm}
\setlength{\oddsidemargin}{0.0cm}
\setlength{\evensidemargin}{-0.3cm}

\pagestyle{plain}

% nouvelles commandes LaTeX, utilis\'ees comme abreviations utiles
\def \be {\begin{equation}}
\def \ee {\end{equation}}
\def \dd  {{\rm d}}

\newcommand{\mail}[1]{{\href{mailto:#1}{#1}}}
\newcommand{\ftplink}[1]{{\href{ftp://#1}{#1}}}
%
% latex SqueletteRapport.tex      % compile la source LaTeX
% xdvi SqueletteRapport.dvi &     % visualise le resultat
% dvips -t a4 -o SqueletteRapport.ps SqueletteRapport % produit un PostScript
% ps2pdf SqueletteRapport.ps      % convertit en pdf

% pdflatex SqueletteRapport.pdf    % compile et produit un pdf

% ======= Le document commence ici ================================

\begin{document}

% Le titre, l'auteur et la date
\title{Squelette d'un rapport en \LaTeX}
\author{Un Nom, Autre Nom\\  % \\ pour fin de ligne
{\small \mail{un.nom@epfl.ch}, \mail{autre.nom@epfl.ch}}}
\date{\today}\maketitle
\tableofcontents % Table des matieres

% Quelques options pour les espacements entre lignes, l'indentation 
% des nouveaux paragraphes, et l'espacement entre paragraphes
\baselineskip=16pt
\parindent=0pt
\parskip=12pt

\section{Préambule}

Un document             \LaTeX{} de type article est subdivis\'e en sections, sous-sections, sous-sous sections.                               
L'utilisateur n'a pas 
besoin de se pr\'eoccupper 
de la  num\'erotation, ni de la police utilis\'ee, ni des espacements.

Si je mets une nouvelle ligne vide, ce qui suit sera dans un nouveau paragraphe.

En effet, 
%\end{document}


\section{Introduction} %------------------------------------------

Bonjour!

\LaTeX{} est un syst\`eme de pr\'eparation de documents de qualit\'e, utilis\'e sp\'ecialement dans les domaines scientifiques et techniques. \LaTeX{} n'est \textbf{pas} un logiciel de traitement de texte.Au contraire, \LaTeX{} incite les auteurs \`a ne \textit{pas} se soucier eux-m\^emes de l'apparence de leurs documents et leur permet de se concentrer sur leur contenu. 

\section{Nouvelle section} %-----------------------------------------------------------

\LaTeX{} fonctionne comme un langage de programmation, dans ce sens que le fichier \textit{source} (ici SqueletteRapport.tex), doit \^etre compil\'e avant de produire un r\'esultat. A la ligne de commande Linux, tapez:

\begin{verbatim}
pdflatex SqueletteRapport.tex
\end{verbatim}
qui d'un coup compile et produit un \verb+.pdf+.

Voir en section\ref{SABC} la remarque concernant le format des fichiers graphiques pour leur inclusion comme figures dans le document.

Nous recommandons d'utiliser Overleaf comme API. Mentionnons qu'il existe d'autres logiciels comme \verb+Kile+ ou \verb+TeXworks+ (freeware) qui int\`grent un \'editeur, un interface graphique utilisateur et un pr\'evisualisateur de pdf.

Il y a plusieurs ressources sur le site Moodle du cours dans le r\'epertoire \verb+Ressources LaTeX+. En particulier, la "feuille de triche" \verb+latexsheet.pdf+ qui r\'esume, sur une feuille, les commandes \LaTeX{} les plus couramment utilis\'ees. La liste compl\`ete des symboles sp\'eciaux et la description des commandes \LaTeX{} qui les produisent se trouvent dans \verb+symbols-a4.pdf+.
   
En guise  d'introduction, on introduira quelques commandes \LaTeX{}. 
Et on corrigera les fautes d'orthograffe. 
Les espaces        et        les         fins                       de    lignes
dans 
le 
fichier \LaTeX{} sont        ignor\'es. Une ligne vide dans le fichier source veut dire qu'un nouveau paragraphe commence dessous. 





Mettre plusieurs lignes vides n'a pas plus d'effet qu'en mettre une seule. 



Les accents grave (g\`ele), aig\"u (d\'ebut), circonflexe (b\^ete), tr\'ema 
sur le i : (na\"\i ve), et la c\'edille s'\'ecrivent comme \c{c}a: 
\begin{verbatim} 
Les accents grave (g\`ele), 
aig\"u (d\'ebut), circonflexe (b\^ete), 
tr\'ema sur le i : (na\"\i ve), et la c\'edille s'\'ecrivent comme 
\c{c}a
\end{verbatim}
%Pour pouvoir taper les accents directement, avec par exemple avec l'application Kile, mettre 
%begin{verbatim}
%Settings > Configure Kile > Editor : Open/Save, onglet General, File Format, Encoding:   iso-8859-1.
%\end{verbatim}
 Dans le pr\'eambule du fichier source (.tex), on mettra:
\begin{verbatim}
\usepackage[T1]{fontenc}
\end{verbatim}

Selon les \'editeurs utilis\'es, selon les syst\`emes d'exploitation, et selon le type de clavier utilisé, il faut parfois changer l'encodage. Par exemple \verb+\usepackage[latin1]{inputenc}+. Impossible ici de faire une liste de toutes les combinaisons...
%Par exemple, avec TeXworks sous Windows, on mettra:
%\begin{verbatim}
% \usepackage[utf8]{inputenc}
%\end{verbatim}

Essai accents :   éàè¨öäü ê île aäö è¨é   ì à ò ù o


Toutes les lignes  plac\'ees entre \verb|\begin{verbatim}| et 
\verb|\end{verbatim}|  apparaitra exactement comme elles 
sont dans le fichier source. Pour faire de m\^eme dans un paragraphe, 
on place le texte entre \verb|\verb|$|$ et  $|$.

Comme le symbole \verb|%| signale que tout ce qui suit est un commentaire et sera donc ignor\'e, il faut, pour \'ecrire \%, le pr\'ec\'eder du symbole \verb|\|, ou comme \verb+\verb|\|+. 
Ainsi le symbole \% sera visible, avec ce qui suit!



%Lorsqu'on veut inscrire les accents directement avec un éditeur, il faut faire attention de 
%choisir l'encodage. Par exemple avec kile, sous Linux, aller sous Settings > Configure > Kile > (panneau de gauche) Editor -> Open/Save, puis sous l'onglet General, s\'electionner Encoding: Unicode (UTF-8). Sauver, puis quitter et rouvrir kile. Ensuite, taper les caractères avec accent.
 
 
 
 
Les \'equations sont soit des expressions ins\'er\'ees dans un paragraphe, par exemple 
$F=ma$, \(E=mc^2\) ou \begin{math} p=mv \end{math}, 
plac\'ees entre \$ et \$ ou entre \verb|\begin{math}| et \verb|\end{math}|, 
soit occupent une ligne s\'epar\'ee, entre \verb|\[| et \verb|\]|,
\[ E=mc^2, \]
ou, avec num\'erotation {\it automatique}, entre \verb|\begin{equation}| 
et \verb|\end{equation}|:
\be
y_{n+1}=y_n+f(y_{n},t_n)\Delta t %g^{\alpha+2}
\ee
\begin{equation}
E=mc^2
\end{equation}
\begin{equation}
\frac{\dd^2y+z-4k+3}{dt^2} = f(y,t)
\end{equation}
Comme on \'ecrira souvent des \'equations, il peut \^etre int\'eressant de d\'efinir de nouvelles commandes. Cela se fait dans le pr\'eambule, c.a.d. avant \verb|\begin{document}|, par exemple:
\begin{verbatim}
\def \be {\begin{equation}}
\def \ee {\end{equation}}
\def \dd  {{\rm d}}
\end{verbatim}
Ainsi  l'\'ecriture s'en trouvera simplifi\'ee: 
\be
\frac{\dd y}{\dd t}=                    f(y,t)      . % Commentaire: en mode \'equation, les espaces du fichier source sont ignor\'es.
\ee

On rajoute des ``d\'ecorations'' sur les symboles, par exemple pour un vecteur 
$\vec{F}=m\vec{a}$, ou $\vec{AB}=\vec{OB}-\vec{OA}$, ou 
$\overrightarrow{AB}$.  

\subsection{R\'ef\'erences crois\'ees} \label{sec:figures} %-------------------

\LaTeX{} a un syst\`eme de r\'ef\'rences crois\'ees pour plusieurs choses. 
Par exemple pour les \'equations. On place \verb|\label{NOMDULABEL}| entre 
\verb|\begin{equation}| et \verb|\end{equation}|.   Soit

\begin{equation} \label{NOMDULABEL}
\vec{F} =m\vec{a}
\end{equation}

\begin{eqnarray}
F &=& ma  \nonumber\\
&+& mc^2 
\end{eqnarray}

On fait r\'ef\'erence \`a cette \'equation avec la commande \verb|\ref{NOMDULABEL}|: de l'Eq.(\ref{NOMDULABEL}), on en tire $F_x=mx''$.

On fait r\'ef\'erence \`a la sous-section \ref{SABC} avec la commande 
\verb|\ref{SABC}|.

% Voici une figure 'flottante', c'est a dire que c'est Latex qui va vous placer
% la figure la ou il lui semble bon.
\begin{figure} %------------------------------------------------
% si on utilise latex, les fichiers graphiques doivent etre au format .eps
% si on utilise pdflatex, ils doivent etre au format pdf, png ou jpg
% Dans l'exemple ci-dessous, tout ce qui est entre \begin{center} et \end{center}
% va être centré horizontalement. Les noms des fichiers sont indiqués entre {} de 
% la commande \includegraphics. Il faut donc avoir dans le même répertoire un fichier 
% de nom correspondant, avec l'extension .png, .pdf ou .jpg si on utilise pdflatex,
% et .eps si on utilise latex. (exemple: aa.png, ohmabellefigure.pdf, ...)
\begin{center}
\includegraphics[width=6cm,angle=0]{ohmabellefigure}
\includegraphics[width=7cm,angle=0]{aa}
\end{center}
% la legende est dans \caption{...}
\caption{\em \label{fig:Plot} % le nom du label est de votre choix
Ceci est une l\'egende. 
}
\end{figure} %---------------------------------------------------

On fait r\'ef\'erence \`a la FIG.\ref{fig:Plot} avec la commande 
\verb|\ref{fig:Plot}|.

Les r\'ef\'erences bibliographiques \cite{Duschmoll_PRL} 
s'obtiennent avec \verb|\cite{Duschmoll_PRL}| 
ou avec \cite{Abi_Science} \verb|\cite{Abi_Science}|.

\subsection{Inclure des figures dans le document} \label{SABC}
%----------------------------------------------------------------------------
Remarque: On a plac\'e un label dans cette sous-section: \verb|\label{SABC}|.
 
Si on compile la source \LaTeX{} avec la commande 
\verb|latex SqueletteRapport.tex|, 
les figures doivent \^etre au format \verb|eps|.

Si on compile avec la commande
\verb|pdflatex SqueletteRapport.tex|, 
les figures doivent \^etre au format \verb|pdf| ou \verb|png| 
ou \verb|jpeg|.

On inclut les figures dans le document dans l'environnement \verb|figure|, entre \verb|\begin{figure}| et \verb|\end{figure}|, avec la commande 

\verb|\includegraphics[width=...cm, ...]{nom_du_fichier} |

Il est mieux de ne PAS mettre explicitement l'extension (.eps ou .pdf ou .png ou .jpg) après le nom du fichier à inclure. latex cherchera aa.eps, pdflatex cherchera aa.pdf ou  aa.png ou aa.jpg

%-----------------------------------------------------------
\section{La structure en sections}
From Mathpix snipping tool.

Eqs du mvmt 3e corps dans $\mathcal{R^\prime}$
$$
\begin{aligned}
\frac{\mathrm{d}^{2}}{\mathrm{~d} t^{2}}\left(\begin{array}{l}
x^{\prime} \\
y^{\prime}
\end{array}\right)=-\Omega^{2} &\left(\begin{array}{ll}
\frac{d^{3} \beta\left(x^{\prime}+\alpha d\right)}{r_{13}^{\prime 3}}+\frac{d^{3} \alpha\left(x^{\prime}-\beta d\right)}{r_{23}^{\prime 3}}-x^{\prime} \\
\frac{d^{3} \beta}{r_{13}^{\prime 3} y^{\prime}}+\frac{d^{3} \alpha}{r_{23}^{\prime 3} y^{\prime}} & -y^{\prime}
\end{array}\right) \\
&+2 \Omega \frac{\mathrm{d}}{\mathrm{d} t}\left(\begin{array}{c}
y^{\prime} \\
-x^{\prime}
\end{array}\right)
\end{aligned}
$$

\subsection{et en sous-sections}

\subsubsection{et en sous-sous-sections}

%\subsubsubsection{et en sous-sous-sous-sections}

%-----------------------------------------------------------
\section{Conclusions}

%-----------------------------------------------------------


\begin{thebibliography}{99}
\bibitem{Duschmoll_PRL} 
 A. Duschmoll, R. Schnok, {\it Phys. Rev. Lett.} {\bf 112} 010015 (2010)
\bibitem{Abi_Science}
 D.J. Abi, {\it et al}, {\it Science} {\bf 22} 1242 (2007)
\end{thebibliography}

\end{document} %%%% THE END %%%%
