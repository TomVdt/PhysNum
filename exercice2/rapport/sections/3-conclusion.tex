\section{Conclusion}
In this report the pendulum with constant and varying length was analysed numerically in great detail. The first results for constant length showed the great precision of the velocity-Verlet method and its good convergence, of order 4 for the position. The obtained results were very coherent with the analytical ones with errors as small as $10^{-18}$. Knowing that this algorithm gives good results it was possible to simulate more complex cases starting with a decreasing length for the pendulum. The order of convergence was affected negatively but this still allowed to see a behaviour coherent with our physical intuition. Despite even the explosion in the angular values the theorem of mechanical energy was verified which gives another indication of the great precision of this numerical method. The last conditions that were analysed were for an oscillating length of the rod. This gave forced oscillators for small oscillations but a chaotic system above a certain treshold of energy. In this chaotic case the linear convergence was not verified anymore and the trajectories became very dependent on the initial conditions. Poincaré maps were used to illustrate the phase space of chaotic and non-chaotic cases which gave a few final informations about the physical behavior of the system.

This study thus allows to confidently use the velocity-Verlet method to simulate oscillators even in non-conventional and chaotic cases. It is also a great tool to determine the characteristics of the oscillations in these cases.