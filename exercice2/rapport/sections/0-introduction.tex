\section{Introduction}

The mass-pendulum system is a well known and often analysed system in classical mechanics. However, some assumptions are often placed on the pendulum, like a constant length or small oscillations in order to explicitly solve the equations of motion. In this report, we will approach this problem numerically, enabling us to study more complex cases. The Velocity-Verlet method is used to carry out the simulations. Along with the study of the system itself, the convergence of the algorithm under different conditions will be determined.

We will first consider the simplest system, with a constant pendulum length. Secondly, the behavior of the mass when the rod is slowly shortened will be analysed. Finally, the length of the rod will vary periodically. The first situation will allow to verify that the simulation remains in line with the theory. The second simulation will allow us to verify the mecanical enegy theorem: lost or gained energy is due to the power of non-conservative forces. Finally, the final setting will allow us to show the chaotic behavior of the system and the overall motion, whether periodic or chaotic, will be shown using a Poincaré map. Some analytical results will also be proven in order to verify the soundness of our simulation by comparing them to the computational results.

The Velocity-Verlet method has been implemented in C++, where the simulations are then carried out. The results of the simulations were analysed using Jupyter Notebooks, which have been included in the extra files associated with this report.