\section{Analytical results}
The subject of this study is waves in shallow waters. The system is considered in only one dimension with the position $x \in [x_L, x_R]$. It is a body of water with depth $h_0(x)$ when at rest. The gravitational acceleration is applied to it with $g = 9.81$ \si{\meter\per\square\second}. With $f(x,t)$ the height of the displacement of water due to the waves we have the equation:
\begin{equation}
    \frac{\partial^2 f}{\partial t^2} = \frac{\partial}{\partial x} \left( g h_0 \frac{\partial f}{\partial x} \right)
    \label{eq:eq1}
\end{equation}

\subsection{Solution for $h_0$ constant}
\underline{Question (a)}
We assume here that we have $h_0(x) \equiv h_0 \in \mathbb{R}$ constant. Thus \autoref{eq:eq1} becomes:
\begin{equation}
    \frac{\partial^2 f}{\partial t^2} = g h_0 \frac{\partial^2 f}{\partial x^2}
    \label{eq:eq_h0_const}
\end{equation}
this is the classical wave equation with speed $u$ such that $u^2 = gh_0$. The general solution for this equation is:
\begin{equation}
    f(x,t) = f_+(x+ut) + f_-(x-ut)
    \label{eq:general_solution}
\end{equation}
with two functions $f_+$ and $f_-$ belonging to $C^2(\mathbb{R})$ and corresponding to different orientations of the wave. Since these two functions take into account the orientation of the wave we can take for the phase velocity $u = \sqrt{gh_0}$ only the positive solution without loss of generality. Then $f_+$ corresponds to a retrograde wave moving along $-\hat{x}$ and $f_-$ to a prograde wave moving along $\hat{x}$.

\subsection{Normal modes}
\underline{Question (b)}
We take the boundary conditions of a fixed left side:
\begin{equation}
    f(x_L, t) = 0 \quad \forall t
    \label{eq:condition_left}
\end{equation}
and a free right side:
\begin{equation}
    \frac{\partial f}{\partial x}(x_R,t) = 0 \quad \forall t
    \label{eq:condition_right}
\end{equation}
and want to find the normal modes and frequencies of this system.

We first take a sine solution of $f_+$ from \autoref{eq:general_solution} as a retrograde wave:
\begin{equation}
    f_+(x,t) = A \sin(kx + \varphi_x + \omega t + \varphi_t)
\end{equation}
with amplitude $A$ and phases $\varphi_x$ and $\varphi_t$ according to position and time. We have here the wavenumber $k$ such that $ku = \omega$. This wave will be reflected inside our basin $[x_L, x_R]$ in a prograde wave. We are searching for stationary waves that have the form $f(x,t) = f_x(x)f_t(t)$ and thus the total displacement as a sum of the prograde and retrograde wave will be
\begin{equation}
    f(x,t) = A(\sin(kx + \varphi_x + \omega t + \varphi_t) + \sin(kx + \varphi_x - \omega t - \varphi_t))
\end{equation}
in order to have a stationary wave:
\begin{equation}
    f(x,t) = 2A\sin(kx + \varphi_x)\cos(\omega t + \varphi_t)
    \label{eq:stationary}
\end{equation}

We take the fixed condition of \autoref{eq:condition_left} with \autoref{eq:stationary} which gives:
\[
    \begin{aligned}
        & 2A\sin(kx_L + \varphi_x)\cos(\omega t + \varphi_t) = 0 \quad \forall t \\
        & \Rightarrow \sin(kx_L + \varphi_x) = 0 \\
        & \Rightarrow \varphi_x = -kx_L + m\pi, \quad m \in \mathbb{Z}\\
    \end{aligned}    
\]
We then have $\forall y, \forall m \in \mathbb{Z}, \sin(y+m\pi) = \pm \sin(y)$ and then $\sin(kx + \varphi_x) = \pm \sin(k(x-x_L))$. This first condition thus gives, by redefining the constant $\pm A$ as $A$:
\begin{equation}
    f(x,t) = 2A\sin(k(x-x_L))\cos(\omega t + \varphi_t)
    \label{eq:sol_first_condition}
\end{equation}

We then take the free side condition of \autoref{eq:condition_right} with \autoref{eq:sol_first_condition} which gives:
\[
    \begin{aligned}
        & 2A k \cos(k(x_R-x_L))\cos(\omega t + \varphi_t) = 0 \quad \forall t \\
        & \Rightarrow \cos(k(x_R-x_L)) = 0 \\
        & \Rightarrow k(x_R-x_L) = \frac{\pi}{2} + n\pi, \quad n \in \mathbb{Z} \\
    \end{aligned}
\]
which allows to define wavenumbers corresponding to normal modes:
\begin{equation}
    k_n = \frac{\pi(2n + 1)}{2(x_R-x_L)}, \quad n \in \mathbb{Z}
    \label{eq:wavenumber_mode}
\end{equation}
Knowing $ku = \omega$ then we have the angular frequencies of the modes:
\begin{equation}
    \omega_n = \frac{\pi(2n + 1)}{2(x_R-x_L)} \sqrt{gh_0}, \quad n \in \mathbb{Z}
    \label{eq:angular_mode}
\end{equation}
Allowing us to rewrite the solution of \autoref{eq:eq_h0_const} for normal modes and the boundary conditions from \autoref{eq:condition_left} and \autoref{eq:condition_right} as:
\begin{equation}
    f_n(x,t) = 2A\sin(k_n(x - x_L))\cos(\omega_n t + \varphi_t)
    \label{eq:normal_mode_simple}
\end{equation}
\begin{equation}
    f_n(x,t) = 2A\sin\left(\frac{\pi(2n + 1)}{2(x_R-x_L)}(x - x_L)\right)\cos\left(\frac{\pi(2n + 1)}{2(x_R-x_L)} \sqrt{gh_0} \, t + \varphi_t\right), \quad n \in \mathbb{Z}
    \label{eq:normal_mode_explicit}
\end{equation}

We can also calculate the frequencies of the stationary waves for each normal mode $n \in \mathbb{Z}$ which are:
\begin{equation}
    \nu_n = \frac{\omega_n}{2\pi} = \frac{(2n + 1)}{4(x_R-x_L)} \sqrt{gh_0}
\end{equation}

In \autoref{eq:normal_mode_explicit} the only two unknown left are $A$ and $\varphi_t$ that can be determined using initial conditions. In this study we will consider $\varphi_t = 0$ such that the initial position of the normal modes $n \in \mathbb{Z}$ are:
\begin{equation}
    f_n(x,t=0) = 2A\sin\left(\frac{\pi(2n + 1)}{2(x_R-x_L)}(x - x_L)\right)
    \label{eq:normal_initial}
\end{equation}


\subsection{WKB analysis for $h_0$ not uniform}
\underline{Question (c)}
We now want to consider a more general case where $h_0(x)$ is not uniform according to $x$. Obtaining analytical solutions in this general case being too complicated a WKB analysis will be done instead. This is shown for \autoref{eq:eq1} in the numerical physics book \cite{physnumbook} (section 4.2.4) and will then be done here for this equation:
\begin{equation}
    \frac{\partial^2 f}{\partial t^2} = g h_0(x) \frac{\partial^2 f}{\partial x^2}
    \label{eq:eq2}
\end{equation}
which does not represent the correct physical situation for $h_0(x)$ not constant.

We first take a general solution in complex exponential form as an ansatz:
\begin{equation}
    f(x,t) = \hat{f}(x)e^{-i\omega t} \quad \mathrm{with} \quad
    \hat{f}(x) = A(x)e^{-iS(x)}
    \label{eq:ansatz}
\end{equation}
and apply it to \autoref{eq:eq2}:
\[
    \begin{aligned}
        \hat{f}(x)(-i\omega)^2e^{-i\omega t} &= gh_0(x)\frac{\partial^2 \hat{f}(x)}{\partial x^2} e^{-i\omega t} \\
        \Rightarrow -Ae^{-iS}\omega^2 &= gh_0 \frac{\partial}{\partial x}\left(\frac{\partial A}{\partial x}e^{-iS} + Ae^{-iS}(-i)\frac{\partial S}{\partial x}\right) \\
        \Rightarrow -Ae^{-iS}\omega^2 &= gh_0 \left(\frac{\partial^2 A}{\partial x^2}e^{-iS} + 2 \frac{\partial A}{\partial x}e^{-iS}(-i)\frac{\partial S}{\partial x} + Ae^{-iS}(-i)^2\left(\frac{\partial S}{\partial x}\right)^2 + Ae^{-iS}(-i)\frac{\partial^2 S}{\partial x^2} \right) \\
    \end{aligned}    
\]
which gives us by simplifying:
\begin{equation}
    -A(x)\omega^2 = gh_0(x)\left(\frac{\mathrm{d}^2 A}{\mathrm{d} x^2}(x) - 2i \frac{\mathrm{d} A}{\mathrm{d} x}(x)\frac{\mathrm{d} S}{\mathrm{d} x}(x) - A\left(\frac{\mathrm{d} S}{\mathrm{d} x}(x)\right)^2 - iA\frac{\mathrm{d}^2 S}{\mathrm{d} x^2}(x)\right)
    \label{eq:equation_exponential_wkb}
\end{equation}
We can do the change $\frac{\partial}{\partial x} \rightleftharpoons \frac{\mathrm{d}}{\mathrm{d}x}$ in the entire equation as every remaining variable depends only on $x$.

We then apply the tagging of each variable depending on how much it varies. We will consider $\varepsilon ^0$ as large values and $\varepsilon^1$ and $\varepsilon^2$ as increasingly smaller values. We have three large values: the depth of the ocean $h_0$, the amplitude $A$ and the rate of change of the phase $S$ according to position. By using the fact that taking the derivative brings to a smaller order of $\varepsilon$ and the notation $'$ for $\frac{\mathrm{d}}{\mathrm{d}x}$ we have:
\begin{equation}
    \begin{array}{rrr}
        A(x) \sim \varepsilon^0 & A'(x) \sim \varepsilon^1 & A''(x) \sim \varepsilon^2 \\
        S'(x) \sim \varepsilon^0 & S''(x) \sim \varepsilon^1 & \\
        h_0(x) \sim \varepsilon^0 & & \\
    \end{array}
\end{equation}
By using this tagging on \autoref{eq:equation_exponential_wkb} we have:
\begin{equation}
    -\mathop{A(x)}_{\varepsilon^0}\omega^2 = g\mathop{h_0(x)}_{\varepsilon^0} \left(\mathop{A''(x)}_{\varepsilon^2} - 2i \mathop{A'(x)}_{\varepsilon^1}\mathop{S'(x)}_{\varepsilon^0} - \mathop{A(x)}_{\varepsilon^0}\mathop{\left(S'(x)\right)^2}_{\varepsilon^0} - i\mathop{A(x)}_{\varepsilon^0}\mathop{S''(x)}_{\varepsilon^1}\right)
\end{equation}
and for each order of $\varepsilon$ ignoring terms of other orders we obtain:
\begin{equation}
    \begin{aligned}
        &\varepsilon^0: &-A(x) \omega^2& = -gh_0(x)A(x)(S'(x))^2 \\
        &\varepsilon^1: &0& = gh_0(x)(-2iA'(x)S'(x) - iA(x)S''(x)) \\
        &\varepsilon^2: &0& = gh_0(x) A''(x) \\
    \end{aligned}
    \label{eq:tagging}
\end{equation}
We neglect the term in $\varepsilon^2$ and we first use the equation in $\varepsilon^0$ to have:
\begin{equation}
    \frac{\omega^2}{gh_0(x)} = (S'(x))^2 \Rightarrow S'(x) = k(x) = \pm \sqrt{\frac{\omega^2}{gh_0(x)}}
    \label{eq:s_prime}
\end{equation}
with $k$ newly defined as $k(x) = \frac{\mathrm{d} S}{\mathrm{d}x}(x)$. We now use the terms in $\varepsilon^1$ from \autoref{eq:tagging} to have the differential equation for $A$:
\[
    \begin{aligned}
        & 2A'(x)S'(x) + A(x)S''(x) = 0 \\
        & \Rightarrow 2\frac{\mathrm{d}A}{\mathrm{d}x}(x)k(x) = - A(x) \frac{\mathrm{d}k}{\mathrm{d}x}(x) \\
        & \Rightarrow \ln(A) = -\frac{1}{2}\ln(k) + C_1, \quad C_1 \in \mathbb{R} \\
        & \Rightarrow A = C_2 k^{-\frac{1}{2}}, \quad C_2>0 \\
    \end{aligned}
\]
We use the formula for $k(x)$ from \autoref{eq:s_prime} and keep only the positive solution to find:
\begin{equation}
    A(x) = C_2 \left(\left(\frac{\omega^2}{gh_0(x)}\right)^{\frac{1}{2}}\right)^{-\frac{1}{2}} = C_2 \left(\frac{gh_0(x)}{\omega^2}\right)^\frac{1}{4} = A_0 (h_0(x))^\frac{1}{4}
\end{equation}
with $A_0$ a constant and thus the amplitude $A \propto (h_0(x))^{1/4}$.

We then have the local wavenumber $k = \left(\frac{\omega^2}{gh_0(x)}\right)^{1/2} \propto (h_0(x))^{-1/2}$ and with $k(x) = \omega / u(x)$ the local phase velocity $u(x)$ is:
\begin{equation}
    u(x) = \frac{\omega}{k(x)} = \sqrt{gh_0(x)} \propto (h_0(x))^{1/2}
\end{equation}
and the local wavelength $\lambda(x)$ is:
\begin{equation}
    \lambda(x) = \frac{2\pi}{k(x)} = \frac{2\pi}{\omega}\sqrt{gh_0(x)}  \propto (h_0(x))^{1/2}
\end{equation}

Our final solution for \autoref{eq:eq2} using the ansatz from \autoref{eq:ansatz} is:
\begin{equation}
    f(x,t) = A(x)e^{-iS(x)}e^{-i\omega t} = A_0 (h_0(x))^\frac{1}{4}e^{-iS(x)}e^{-i\omega t}
\end{equation}
with $S(x) = \int S'(x) \mathrm{d}x$ that can be found using \autoref{eq:s_prime} when knowing the formula for $h_0(x)$.


