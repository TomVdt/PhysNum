
% Something figure spacing
% \setlength\intextsep{5pt}

% Smallcaps autoref
\let\oldautoref=\autoref
\renewcommand{\autoref}[1]{\textsc{\oldautoref{#1}}}

% Legendes correctes
% \captionsetup[figure]{name=Figure}
% \captionsetup[table]{name=Tableau}

% Template de mes couilles
\makeatletter
\renewcommand{\section}{\@startsection {section}{1}{\z@}
	{-3.5ex \@plus -1ex \@minus -.2ex}%
	{2.3ex \@plus.2ex}%
	{\normalfont\normalsize\bfseries}}
\makeatother

\makeatletter
\renewcommand{\subsection}{\@startsection {subsection}{1}{\z@}%
	{-3.5ex \@plus -1ex \@minus -.2ex}%
	{2.3ex \@plus.2ex}%
	{\normalfont\normalsize\bfseries}}
\makeatother

% Ancien titre pour les rapports de labo
% \newcommand{\couille}[5]{
%	\begin{center}
%		\large\textbf{\sffamily Expérience #1: #2}\\%
%		\large\sffamily Groupe N$^\circ$#3: #4\\%
%		\large\sffamily \today\qquad #5\\%
%	\end{center}
%}


\newcommand{\mail}[1]{{\href{mailto:#1}{#1}}}

\newcommand{\couille}[5]{
	\title{#1}
	\author{#2, #3\\  % \\ pour fin de ligne
	{\small \mail{#4@epfl.ch}, \mail{#5@epfl.ch}}}
	\date{\today}\maketitle
	\tableofcontents % Table des matieres

}

\def \dd  {\ensuremath{\mathrm{d}}}