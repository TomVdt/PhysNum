\section{Introduction}
It is crucial for many applications in the space domain to calculate orbits accurately in order to know exactly where objects will stand at a given moment. This can then be used to calculate trajectories, gravity assists or required propulsions to reach a certain orbit. This is however a complicated task as often orbits have points where numerical errors quickly add up significantly. It is therefore important to be able to simulate very precisely in certain areas and reduce computation time by simulating less precisely in other parts. In this report simulations of orbit to and around the L2 Sun-Earth Lagrange point will be presented to represents the trajectories of the James Webb Space Telescope (JWST or Webb). The simulations will be run using the Runge-Kunta order 4 (RK4) method with fixed and adaptative time steps.

The algorithms used will be implemented through a C++ simulation and the obtained data
analysed using classic python libraries. The first set of simulation will be from a low orbit around Earth to the position of the L2 Lagrange point. The second point of interest will be the trajectory in the long run of Webb once it has reached L2 which is a point of unstable equilibrium. Some analytical results will also be proven in order to verify the soundness of our simulation by comparing them to the computational results.